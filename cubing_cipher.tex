\documentclass[a4p]{jarticle}
\usepackage{abstract}
\usepackage[dvipdfmx]{graphicx}
\usepackage{amsmath}

\begin{document}

\thispagestyle{empty}
\begin{center}
\vspace*{4.5cm}
{\huge\bf 新規暗号の提案及び安全性評価}
\vspace*{3cm}

{\large 2019年11月X日}
\vspace*{3cm}

{\large 立命館大学 情報理工学部}
\vspace*{5mm}

{\Large 石~川~~琉~聖}
\end{center}
\newpage 

\setcounter{page}{1}

\section*{概要}

本論文では,新規暗号であるcubing暗号を提案し,その安全性評価を行った.本論文の貢献は新たな安全な暗号の選択肢を増やしたことである.暗号の安全性の評価については,適切な鍵長の計算,ブルートフォース攻撃対策,頻度分析,選択平文攻撃,replay攻撃について分析した.
分析の結果,鍵長は50文字$=50Byte$が良く,またいずれの攻撃手法においてもcubing暗号に対しては有効ではないことがわかった.

\newpage

\tableofcontents
\clearpage
\section{残っている作業}
概要(もう少し書きたい)\\
IVを計算によって求めたい\\
\section{スケジュール}
\begin{table}[hbtp]
\begin{tabular}{|l|l|l|}
\hline
いつまで & 内容 & 備考\\ \hline
11/16 & 参考文献と関連研究 & \\ \hline
11/23 & まとめ & \\ \hline
11/30 & \begin{tabular}[c]{@{}l@{}}二段組\\ IVの計算方法\\ 概要\\ 参考文献を左揃え\end{tabular} & \\ \hline
12/7  & 400字の発表内容など執筆 & \begin{tabular}[c]{@{}l@{}}昨年度の記載内容:\\1)発表者情報(氏名、所属、e-mail、電話番号)\\ 2)発表タイトル\\ 3)発表概要(400字程度)\\ 4)自己アピール\\(情報セキュリティに関連した研究、開発、サービス運用、\\コミュニティ運営、インシデント対応、脆弱性発見等\\があれば記述してください。)\end{tabular} \\ \hline
12/13 & できれば初日に提出 & 昨年度の募集開始日\\ \hline                        
\end{tabular}
\end{table}

\twocolumn

\section{はじめに}
%暗号がここらへんで使われててそこそこ大事って言う社会的背景を語る

インターネットの普及により,研究機関のみならず個人もインターネットを使うようになった.それに伴い,個人情報や社外秘情報などがインターネット上で取引され,情報の秘匿性が重要視されるようになった.それら秘匿情報を第三者に見られないようにするため,インターネット上では暗号技術というものが使われている.
暗号技術とは特定の情報を隠蔽する技術で,その代表的なものにシーザー暗号やエニグマ暗号などがある.現在頻繁に使われる暗号としてRSA暗号やAES暗号があり,インターネット上で殆ど全ての秘匿情報を扱う際に利用されている.\\
本論文では新しい暗号cubing暗号を提案する.\\

\section{本論}
\subsection{cubing暗号の概要}
cubing暗号の概要は以下の通りである.なお,本暗号に関するソースコードや素材などは全てgithubに載せられている.\cite{github}

\begin{itemize}
  \item 平文ブロック長:45byte
  \item 暗号文ブロック長:108byte
  \item 鍵長:可変
  \item 暗号化処理の並列化:可 (※但し,Shuffle処理がボトルネックとなる)
  \item 復号処理の並列化:可 (※但し,Sort処理がボトルネックとなる)
  \item Random Read:不可
\end{itemize}
なお,cubing暗号は暗号利用モード「cubingmode(CBGモード)」の利用を前提とし,以下cubing暗号に対しcubingmodeを利用した際の手順を説明する.
\subsection{cubing暗号の暗号化・復号手順}

以下の順序で暗号化・復号を行う.

\subsubsection{暗号化}
\begin{enumerate}
\item 平文を用意する.
\item 必要があればパディング処理を行う.
\item encrypt処理を行う.
\end{enumerate}

\subsubsection{復号}
\begin{enumerate}
\item decrypt処理を行う.
\item パディング処理を逆に行う.つまり,暗号化の際に付属した文字を取り除く.
\item 平文を取得する.
\end{enumerate}

\subsubsection{パディング処理}
平文ブロックが45Byteに満たない場合,1Byte分null入れ,残りはランダムな英数列を入れる.これにより,平文を45Byteに固定することができる.

\subsubsection{コンピュータ上での表現方法}
ルービックキューブをコンピュータ上で表現する際には配列を用いる.具体的には以下のルービックキューブの展開図と配列の添字が対応する.\\

%\begin{center}
  \includegraphics[width=7cm]{./tex_pic/seq.jpg}\\
%\end{center}
    
\subsubsection{転置の仕様}
cubing暗号転置処理は方向・列・回数の三つより決定される.方向に関しては以下の図のように三種類定義される.縦方向はルービックキューブで処理する際,以下の図のように回転させる方向のことである.なお,図には各方向に対する列が記載されている.
\begin{center}
  \includegraphics[width=7cm]{./tex_pic/tate.jpg}\\
\footnote{http://iamthecu.be/}
\end{center}
横方向に関しては以下のように定義されている.
\begin{center}
  \includegraphics[width=7cm]{./tex_pic/yoko.jpg}\\
\footnotemark[1]
\end{center}
回転方向に関しては以下のように定義されている.
\begin{center}
  \includegraphics[width=7cm]{./tex_pic/kai.jpg}\\
\footnotemark[1]
\end{center}

なお,cubing暗号の転置はルービックキューブの転置とは一部違った箇所がある.それは,「ある方向のある列を転置させる際,他の列およびマスは一切転置させない」ということである.具体的には以下の図のような転置がなされる.\\
\begin{center}
  \includegraphics[width=9cm]{./tex_pic/trans.png}\\
\end{center}
まず「cubing暗号の転置操作」で黄色のマスにっているのは鍵(縦方向1列1回)転置操作で場所が入れ替わったマスを表している.一方,一般的なルービックキューブの転置操作では同じ鍵でgubing暗号に加えてさらに転置するマスを黄色く色づいている.一見一回の操作で多くのマスを転置できるように見えるが,実際はルービックキューブの角の3マスや,側面の2マス常に隣り合った関係を持ってしまう.よって全探索や頻度分析の手掛かりになってしまうと考えた.したがって,このような転置はしないようになっている.

\subsubsection{暗号文の取得}
最後に,暗号文は以下の図の順で取り出す.
\begin{center}
  \includegraphics[width=7cm]{./tex_pic/seq.jpg}\\
\end{center}
\subsection{cubingmodeの利用手順}

以下の順序で暗号化・復号を行う.
\subsubsection{暗号化}
\begin{enumerate}
\item 平文をブロックに分ける.
\item 必要があればパディングを用意する.
\item 全ての平文に対してマスク処理を行う.
\item 全てのブロックに対し,エンコードを行う.
\item 全てのブロックの平文以外の箇所に対してマスク処理を行う.
\item ブロックごとに暗号化を行う.
\item ブロックをシャッフルする.
\end{enumerate}

\subsubsection{復号}
\begin{enumerate}
\item 暗号文をブロックに分ける.
\item 全てのブロックに対して復号を行う.
\item 暗号化の(5)で使用したマスク処理を元に戻す.
\item (3)により現れるシーケンス番号を元に,ブロック間ソートを行う.
\item 暗号化の(3)で使用したマスク処理を元に戻す.
\item 全てのブロックに対し,デコードを行う.
\item 全てのブロックを結合させ,平文を生成する.
\end{enumerate}

\subsubsection{cubingmodeの各処理の適用範囲}
各ブロックの内容は以下の通りである.また,mask(1)とmask(2)処理の適用範囲に関しても以下の通りである.
\begin{center}
  \includegraphics[width=7cm]{./tex_pic/block.png}\\
\end{center}

\subsubsection{62進数の説明}
cubingmodeのエンコード処理内では62進数が使われる.62進数は,「AA, AB, AC, ... , AZ, Aa, Ab, Ac, ... , A0, A1, A2, ... , A9, BA, BB, ... , B9, CA, ... , 99, ...」の順で定義される.なお,本暗号利用モードではAA~99までが使用される.

\subsubsection{エンコード処理}
本節では以下の記号を用いる.

\(n\):現在操作しているブロックの番号(0,1,2,...)\\
\(C_i\):ブロックの\(i\) Byte目の文字\\
\(\mathrm{toas}(C)\):特定の文字\(C\)のASCIIコードにおける番号\\
\(\mathrm{fras}(N)\):特定の数字\(N\)のASCIIコードにおける文字\\
\(\mathrm{rand}()\):アルファベット又は数字から1文字を一様に選ぶ関数\\
\(\mathrm{to62}(x)\):0から61までの自然数xを62進数に従って文字にする関数

\footnotesize
\begin{align}
C_{45+i} &= \mathrm{fras}\left(\left(\left(\sum_{j=9(i-1)+1}^{9i}{\mathrm{toas}(C_j)}\right)\bmod 26\right)+97\right) \notag\\
C_{51} &= \mathrm{to62}\left (\left \lfloor\frac{n}{62}\right \rfloor \right) \notag\\
C_{52} &= \mathrm{to62}\left ( n \bmod 62 \right) \notag\\
C_{53} &= \mathrm{rand}() \notag\\
C_{54} &= \mathrm{rand}() \notag\\
\end{align}
\normalsize

とする.ただし,\(i=1,2,3,4,5\)とする.これを全てのブロックに対して行う.

\subsubsection{表示可能文字}
以下の配列で定義される.
\begin{verbatim}
printable_table[] = {'0', '1', '2', '3', 
'4', '5', '6', '7', '8', '9', 'a', 'b', 
'c', 'd', 'e', 'f', 'g', 'h', 'i', 'j', 
'k', 'l', 'm', 'n', 'o', 'p', 'q', 'r', 
's', 't', 'u', 'v', 'w', 'x', 'y', 'z', 
'A', 'B', 'C', 'D', 'E', 'F', 'G', 'H', 
'I', 'J', 'K', 'L', 'M', 'N', 'O', 'P',
'Q', 'R', 'S', 'T', 'U', 'V', 'W', 'X', 
'Y', 'Z', '!', '"', '#', '$', '\%', '&', 
'\'', '(', ')', '*', '+', ',', '-', '.', 
'/', ':', ';', '<', '=', '>', '?', '@', 
'[', '\\', ']', '^', '_', '`', '{', '|', 
'}', '~', ' ', '\n', '\0', '\t'};
\end{verbatim}

この順序はpythonのstring.printableを元に作られた.エスケープシーケンスに関してはASCIIコードに対応している.

\subsubsection{mask(1)処理}
mask(1)では各ブロックの平文45Byteにマスク処理を行う.まず,以下を定義する.\\
平文の\(i\)文字目:\(Pi\)\\
\(i\)番目のマスク:\(Mi\)\\
\(\mathrm{printable\_table}\)内の文字\(Pi\)のインデックス:\(\mathrm{Idx}(Pi)\) (※\(\mathrm{printable\_table}\)とは,上記「表示可能文字」のこと)\\
とすると,


\[Pi=\mathrm{printable\_table}[(\mathrm{Idx}(Pi)+Mi) \bmod 98]\\\]


と定義する.ただし,\(i=1〜45\)とする.

\subsubsection{mask(2)処理}
mask(2)ではハッシュとシーケンス番号にマスクをかける.計算方法は上記「mask(1)」と同様.

\subsubsection{encrypt処理}
encrypt処理ではcubing暗号の暗号化処置(転置処理)を行う.

\subsubsection{decrypt処理}
decrypt処理ではcubing暗号のencrypt処理を元に復号処理を行う.ルービックキューブは同じ方向・同じ列で4回回転させると元に戻る性質があるため,暗号化に回転した回数を\(n\)とおくと,復号処理では\(4-n\)回転置処理を行う.

\subsubsection{shuffle処理}
暗号化後に全ブロックをブロックごとにシャッフルする.Fisher–Yatesのアルゴリズムを用いることによって高速に実現可能となる.

\subsubsection{sort処理}
各ブロックの51,52番目の文字が昇順になるようにブロックごとにソートする.

\subsubsection{送信内容}
送信内容は以下の三つで構成され,三つが連結された状態で送信される.

1. 暗号文
2. mask(1)
3. mask(2)
ただし,mask(2)に関してはどのmask(2)がどのブロックに適用されるのか判別するため,各ブロックの後ろに記述する.
例) [block1] [mask(2)-1] [block3] [mask(2)-3] [block2] [mask(2)-2]

\subsubsection{ブロック数が多い時の対策}
\(62^2\)のブロックを1つの大きなブロックとして考え,各大きなブロックをECBモードで暗号化する.
上記「shuffle」では,\(62^2\)個の大きなブロックごとにシャッフルする.

\section{評価実験}
ここでは主に本暗号・暗号利用モードの長所を実験結果をもとに記す.
\subsection{推奨する鍵の長さに関して}
鍵の長さを変えた際,各マスが転置されている割合を計算した.\\

\subsubsection{数学的アプローチ}
1回の転置処理でfrac{12}{54}文字変化する.つまり,1回の転置処理でfrac{42}{54}文字は変化しない.\\
ここで$n=$転置処理の回数とする.すると,特定の文字が変化しない確率は
\[\left(\frac{42}{54}\right)^n\]
で計算できる.ここで,全ての文字の変化する確率と変化しない確率を等しくするためには,以下の式が成り立つ必要がある.
\[\left(\frac{42}{54}\right)^n \leq \frac{1}{54}\]

これを満たす最小の$n$は16で,この時の左辺の値は約0.0179である.また,鍵長$=$転置回数$\times 3$なので,鍵長は$16*3=48$文字以上を推奨する.

\subsubsection{機械的アプローチ}

鍵長を$1$から$300$まで検証した.各鍵長に対して100回暗号化処理を行い,転置されていない文字がどれくらいあるのか,それぞれ平均値を計算した.\\
なお,計算に用いたプログラムはkey\_len.cpp,計算結果はkeylen.csv,グラフの描画はkeylen\_print.pyである.\\
\begin{center}
  \includegraphics[width=8cm]{./tex_pic/figure.png}\\
\end{center}
上図より,鍵長は約50から値が変わらない,ということが読み取れる.よって鍵長は約50文字以上である必要がある.

\subsection{暗号化に要する時間}
1秒間に約18Mbyteの暗号化が可能である.これはtimecalc.cppによって計算した.

\subsection{攻撃対策}
\subsubsection{ブルートフォース攻撃}
鍵長がわからないことから,存在し得る平文は$54!$通りある.ここで,「$!$」は階乗記号を表す.\\
ただし,encrypt処理の性質上,ルービックキューブの各面に対して各マスは以下のように分類される.
\begin{table}[htb]
  \begin{tabular}{|l|c|r|} \hline
    1 & 2 & 1 \\ \hline
    2 & 3 & 2 \\ \hline
    1 & 2 & 1 \\ \hline
  \end{tabular}
\end{table}
これはencrypt処理をした際に転置可能なマスを集合を表している.例えば1と書かれたマスはどのような鍵を使っても1と書かれたマスにしか転置できない.これを利用するとブルートフォース攻撃をする際のパターン数が減少する.\\
集合1に存在するマスは$4*6=24$マス,集合2に存在するマスは$4*6=24$マス,集合3に存在するマスは$1*6=6$マス,よって存在し得る平文は $24!*24!*6!$通りある.\\
ここで,コンピュータ以下の条件を満たすとする.
\begin{enumerate}
  \item 生成した平文が正当なものなのか判定するのに時間がかからない
  \item マスク処理,ソート処理など暗号化・復号にかかる全ての処理に時間がかからない
  \item 暗号文を並びかえた文字列を1秒間に$10^9$回生成できる
\end{enumerate}
この時,ブルートフォース攻撃にかかる最大の時間は$\dfrac{(24!)^2\times6!}{10^9}$秒である.\\
これは$2.77\times10^{50}$秒であり,約$8.79\times10^{42}$年である.よって現実的な時間での解読は不可能である.

\subsubsection{頻度分析}
暗号文の各文字はマスク処理を施されたランダムな文字またはエンコード処理のrand()計算で生成されたランダムな文字である.つまり全ての文字の出現頻度は等しくなる.よって頻度分析は不可能である.また,仮に攻撃者によってマスクが外されたと仮定する.しかし,コンピュータ上で平文と予測した文字列が正しい平文であるという判定ができない.
\subsubsection{選択平文攻撃}
マスク処理で使用されるマスクは乱数である.これは暗号化する度に生成されるため,同じ平文を二度暗号化しても同じ暗号文にならない.よって選択平文攻撃は不可能である.
\subsubsection{replay攻撃}
暗号化の最後にシャッフル処理を行う.これによりどのブロックがどこに存在するのか,秘密鍵がわからない第三者にはわからない.よってreplay攻撃は基本的に成立しない.ただし,ブロック数を$N$,すり替えたいブロック数を$M$とすると,$\dfrac{1}{N^M}$の確率でreplay攻撃が成立してしまう.しかし,Nは小さくないことを想定している為,replay攻撃は実用性に欠けると考える.

\section{関連研究}
\subsection{“RUBIK'S CUBE” AS A TRANSPOSITION DEVICE}
この研究ではルービックキューブを用いた暗号について提案されている.\cite{Mitchell}暗号化の手順は以下の通りである.
\begin{enumerate}
  \item ルービックキューブの6つの面にそれぞれ1〜6まで数字を書き込む.ただし,1だけは面の左上のマスに書き込む.
  \item 面同士の入れ替えを行う.ただし,1が書かれている面に関しては入れ替えを行ってはならない.この時,並び替えの場合の総数は$5!=120$通り存在する.
  \item 各面に平文を8文字ずつ,数字が書かれていないマスに書き込む.
  \item ルービックキューブを回転させる.ただし,左上にある1の数字だけは回転させてはならない.
  \item スペースはランダム性のため,好きな数nullに置き換えても良い.
  \item 出来たルービックキューブの各面に書かれている数字の順に文字を取り出す.この時の文字が暗号文になる.
\end{enumerate}

この暗号はThe Rubik's Crypto-Cube: a Trans-Composite Cipher\cite{Trans-Composite Cipher}によって解読されている.Mitchell Cryptsystemsの弱点は主に二つある.一つはルービックキューブの角にある3マスは常に同じ位置関係にあること,二つ目は暗号文が平文の中にあることである.
本論文のcubing暗号はこれらの弱点に対して対策を行った.


\subsection{Scrambling algorithm for encryption of text using cube rotation artificial intelligence technique}

こちらの論文\cite{Scrambling algorithm}でもルービックキューブを用いた暗号が使用されている.但し,この論文ではルービックキューブの回転操作を人工知能を用いている.この論文では暗号文は平文と比べ完全にスクランブルされていると書かれている.しかし,暗号文の中に平文の文字が入った状態となってしまう他,改竄検知ができない.そこで,cubing暗号ではこれらに関して対策を行った.


\section{まとめ}

\subsection{本論文の内容}
本論文では自作の暗号cubing暗号についての定義とその安全性について述べた.ルービックキューブを用いた転置暗号は既に開発されていたが,私はそれらの暗号を弱点を踏まえて解かれにくい暗号の作成を行った.解かれにくい暗号というのは,安全性が保証された条件下で様々な攻撃に対して無効である暗号である.

\subsection{推奨される鍵長に関して}
暗号の安全性が保証される鍵長に関しての検証も行った.検証は平文に転置処理を施して元の平文との転置率を機械的に求める方法と数学的に求める方法で二度行った.その結果,どちらも鍵長は50文字以上が好ましいとわかった.

\subsection{攻撃対策に関して}
本論文では主要な攻撃手法であるブルートフォース攻撃・頻度分析・選択平文攻撃・replay攻撃に関して,暗号の安全性の検証を行った.その結果,ブルートフォース攻撃に対しては計算量が指数時間になるため,実用的ではないと判断し,頻度分析に対しては暗号文がマスクされている上,たとえマスクが外されてもコンピュータ上で平文と予測した文字列が正しい平文であるという判定ができない.よって頻度分析も有用ではない.そして,選択平文攻撃に対してはマスク処理が施されており,マスクによって暗号文の各文字はランダムな値になる.最後に,replay攻撃に対しては,暗号文ブロックはシャッフルされるので,replay攻撃が成立する確率はブロック数を$N$,すり替えたいブロック数を$M$とすると,$\dfrac{1}{N^M}$である.これは実用的ではない.\\\\
以上の結果から本暗号は以前から存在するルービックキューブによる転置暗号の脆弱性を対策し,より安全な暗号の制作に成功した.

\section{謝辞}
本研究を進めるに当たり、サイボウズ・ラボ株式会社 緑川 志穂氏からは多大な助言を賜り,また丁寧に指導して下さいました。厚く感謝を申し上げます。また暗号の解読やアルゴリズム修正等を快く協力して頂いたセキュリティ・キャンプ2019集中開発コース標準ゼミのメンターの皆さま,及び様々な助言を頂きました受講生の皆さまにも感謝の意を表します.

\newpage
\begin{flushleft}
\begin{thebibliography}{}

\bibitem{github} Githubレポジトリ : https://github.com/xryuseix/cubing\_cipher-SecCamp2019-

\bibitem{Mitchell} “RUBIK'S CUBE” AS A TRANSPOSITION DEVICE : https://www.tandfonline.com/doi/pdf/10.1080/0161-119291866928?needAccess=true

\bibitem{Trans-Composite Cipher}The Rubik's Crypto-Cube : a Trans-Composite Cipher : https://epublications.regis.edu/cgi/viewcontent.cgi?article=1511\&context=theses

\bibitem{Scrambling algorithm}Scrambling algorithm for encryption of text using cube rotation artificial intelligence technique : https://pdfs.semanticscholar.org/1a63/be53b7ddc422107a9f4343eaa6cfe6732a70.pdf

\end{thebibliography}
\end{flushleft}
\onecolumn 
\end{document}
